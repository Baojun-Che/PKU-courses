\documentclass[12pt, a4paper, oneside]{ctexart}
\usepackage[margin=1in]{geometry}
\usepackage{graphicx}
\usepackage{animate}
\usepackage{subcaption}
\usepackage{url}
\usepackage{amssymb}
\usepackage{amsmath}
\usepackage{multirow}
\usepackage{listings}
\usepackage{algorithm}
\usepackage{algpseudocode}
\renewcommand{\algorithmicrequire}{\textbf{输入:}}
\renewcommand{\algorithmicensure}{\textbf{输出:}}
\usepackage{booktabs}
\usepackage{float}
\usepackage{caption}
\usepackage[table,xcdraw]{xcolor}

\usepackage[colorlinks,linkcolor=blue]{hyperref}

% 代码高亮设置
\lstset{
    columns=fixed,
    frame=none,
    keywordstyle=\color[RGB]{40,40,255},
    numberstyle=\footnotesize\color{darkgray},
    commentstyle=\itshape\color[RGB]{0,96,96},
    stringstyle=\rmfamily\slshape\color[RGB]{128,0,0},
    showstringspaces=false,
    tabsize=4
}

\pagestyle{empty}

% 自定义引用命令
\newcommand{\figref}[1]{图\ref{#1}}
\newcommand{\tabref}[1]{表\ref{#1}}
\newcommand{\equref}[1]{式\ref{#1}}
\newcommand{\secref}[1]{第\ref{#1}节}

\begin{document}

\title{\textbf{Group LASSO Problem 编程作业报告}}
\author{车保骏 2501110040}
\maketitle

\section{问题描述}
算法需要求解的 Group LASSO 问题为:
\begin{equation}\label{eq:prob}
    \min _{x \in \mathbb{R}^{n \times l}} \frac{1}{2}\|A x-b\|_F^2+\mu\|x\|_{1,2}
\end{equation}
其中 $A \in \mathbb{R}^{m \times n}$, $b \in \mathbb{R}^{m \times l}$, $\mu>0$, 
$$
\|x\|_{1,2}:=\sum_{i=1}^n\|x_{i,:}\|_2,
$$
其中 $x_{i,:}$ 是矩阵 $x$ 的第 $i$ 行。

\section{测试数据生成}
本文使用算法~\ref{al:data_init} 生成测试数据,取 \textbf{numpy.random.seed 随机种子为 97006855},取 $n = 512$, $m = 256$, $l = 2$, $\mu = 0.01$, $sparse = 0.1$.

\begin{algorithm}[htbp]
\caption{生成测试数据的伪代码}
\label{al:data_init}
\begin{algorithmic}[1]
\Require 种子 $seed$, 维度 $n$, $m$, $l$, 正则项系数 $\mu$, 稀疏度 $sparse$
\Ensure 矩阵 $A$, 真实解 $u$, 观测值 $b$, 初始值 $x_0$

\State 设置随机种子: \texttt{np.random.seed(seed)}
\State 计算非零元素个数: $k \gets \mathrm{round}(n \times sparse)$
\State 生成随机矩阵: $A \gets \texttt{np.random.randn}(m, n)$
\State 随机选择支撑集: $p \gets \texttt{np.random.permutation}(n)[:k]$
\State 初始化零矩阵: $u \gets \texttt{np.zeros}((n, l))$
\State 在支撑集上赋值: $u[p, :] \gets \texttt{np.random.randn}(k, l)$
\State 计算观测值: $b \gets A u$
\State 生成初始值: $x_0 \gets \texttt{np.random.randn}(n, l)$
\State \Return $A, u, b, x_0$
\end{algorithmic}
\end{algorithm}

\section{\textbf{评价指标及结果总览}}

项目完成了多个算法的实现,为了实现不同结果效率之间的对比,本项目将使用以下指标来评价各个算法的性能:

\begin{itemize}
    \item \textbf{Fval}: 算法得到的最终目标函数值
    \item \textbf{Errfun}: 算法的解 $x$ 和 cvx-mosek 的解 $x^*$ 的相对距离 $ \dfrac{\Vert x-x^* \Vert _F}{1+\Vert x^* \Vert_F} $
    \item \textbf{ErrExact}: 算法的解 $x$ 和用于生成数据的 $u$ 的相对距离 $ \dfrac{\Vert x-u\Vert _F}{1+\Vert u\Vert _F} $
    \item \textbf{Time}:算法运行时间(单位:秒)
    \item \textbf{Iter}:算法迭代次数\footnote{对于增广拉格朗日法解对偶问题 (ALM-dual),由于内循环中不更新 $x$,所以迭代次数指外循环的迭代次数。}
    \item \textbf{Sparsity}:算法得到的最终解的稀疏度,使用
    \[
        \text{Sparsity} = \frac{1}{nl}\#\{(i,j): |x_{ij}|>10^{-6}\cdot\max_{k,l}|x_{kl}| \}
    \]
    计算,数值越小说明越稀疏。
\end{itemize}

本文涉及的所有算法在上述评价指标下的表现见表~\ref{table:all}。其中,由于 \textbf{Errfun} 以 cvx-mosek 为参考,所以 cvx-mosek 一行未标出 \textbf{Errfun};调用 mosek 或 gurobi 求解器的算法均无法获得迭代次数。此外,以 cvx-mosek 的最优值 $f^*$ 为参考,我们还计算了部分算法迭代中相对误差 $\dfrac{f^k-f^*}{f^*}$ 的变化,见图~\ref{fig:all}。

\begin{table}[htbp]
\centering
\begin{tabular}{l|r r r r r r }
\hline
\textbf{Solver} & \textbf{Fval} & \textbf{Errfun} & \textbf{ErrExact} & \textbf{Time(s)} & \textbf{Iter} & \textbf{Sparsity} \\
\hline
cvx\_mosek       & 0.67057522 & --        & 3.56e-04    & 1.8220   & --    & 0.1035 \\
cvx\_gurobi      & 0.67057522 & 7.66e-06  & 3.61e-04    & 2.1429   & --    & 0.1055 \\
mosek\_direct    & 0.67057522 & 4.33e-07  & 3.56e-04    & 0.4263   & --    & 0.1025 \\
gurobi\_direct   & 0.67057522 & 1.66e-06  & 3.57e-04    & 0.9032   & --    & 0.1025 \\
SGD\_primal      & 0.67059517 & 6.93e-04  & 9.66e-04    & 4.7122   & 2357  & 0.2852 \\
GD\_primal       & 0.67058665 & 3.30e-04  & 6.53e-04    & 2.7558   & 1381  & 0.5156 \\
ProxGD\_primal   & 0.67057599 & 1.26e-04  & 4.65e-04    & 0.1868   & 643   & 0.1357 \\
FProxGD\_primal  & 0.67057528 & 3.22e-05  & 3.37e-04    & 0.1202   & 253   & 0.1025 \\
ALM\_dual        & 0.67058907 & 3.71e-04  & 6.71e-04    & 0.7970   & 9     & 0.0996 \\
ADMM\_dual       & 0.67058733 & 3.33e-04  & 6.44e-04    & 0.0704   & 70    & 0.0996 \\
ADMM\_primal     & 0.67060069 & 1.62e-04  & 4.46e-04    & 0.2093   & 569   & 0.6055 \\
\hline
\end{tabular}
\caption{优化算法性能比较}
\label{table:all}
\end{table}

\begin{figure}[htbp]
    \centering
    \includegraphics[width=0.85\linewidth]{figs/plot_all_solvers.pdf}
    \caption{不同算法迭代中相对误差 $\dfrac{f^k-f^*}{f^*}$ 的变化}
    \label{fig:all}
\end{figure}

\section{调用求解器求解}

\subsection{CVX 调用 Mosek 和 Gurobi 求解器}
由表~\ref{table:all} 可见,使用 CVX 调用 Mosek 和 Gurobi 求解器的效率相当。两者在最优值上的计算结果几乎相同,在运行时间和解的稀疏度上也很相近。

\subsection{直接调用 Mosek 和 Gurobi 求解}
为了使原问题能直接被 Mosek 和 Gurobi 求解(不使用 CVXPY),我们引入辅助变量 $t$,构造二阶锥规划。等价的 SOCP-QP 模型为:
\[
\begin{aligned}
\min_{x \in \mathbb{R}^{n l},\, t \in \mathbb{R}^n} \quad & \frac{1}{2} \|A x - b\|_F^2 + \mu \sum_{i=1}^n t_i \\
\text{s.t.} \quad & \| x_{i,:} \|_2 \leq t_i, \quad i = 1,\dots,n.
\end{aligned}
\]
这就可以直接调用 Mosek 和 Gurobi 求解。表~\ref{table:all} 的结果显示,直接调用两种求解器的求解速度明显快于借助 CVX,且获得的解的稀疏性稍高。相对而言,直接调用 Mosek 比直接调用 Gurobi 求解的速度更快。

\section{其它算法}

\subsection{连续化策略}
对于 Group Lasso 原始问题 (\ref{eq:prob}),我们采用连续化策略:从一个较大的正则化参数 $\mu_t$ 开始,逐步减小至目标值 $\mu$,并在每个阶段求解对应的 LASSO 问题:
\begin{equation}\label{eq:prob-mu}
    \min _{x \in \mathbb{R}^{n \times l}} f_{\mu_t}(x):=\frac{1}{2}\|A x-b\|_F^2+\mu_{t}\|x\|_{1,2}.
\end{equation}
该策略的优势在于:在求解当前 $\mu_t$ 对应的问题时,可以将前一阶段(对应 $\mu_{t-1}$)得到的最优解作为高质量的初始点(仅对第一个子问题 $\mu_1$ 使用随机初始点)。由于较大的 $\mu_t$ 会使问题具有更强的正则性,解更稀疏且优化问题更易求解。通过这种方式,连续化策略将原始困难问题转化为一系列由易到难的子问题,利用前序解提供良好初值,从而显著加速整体求解过程。

在本项目中,SGD 算法、GD 算法、ProxGD 算法和 FProxGD 算法使用连续化策略,具体结果见下面的小节。

\subsection{原问题的对偶问题}
对于 Group Lasso 原始问题 (\ref{eq:prob}),可以考虑通过求解其对偶问题求解原问题。下面做一下对偶问题的推导。首先引入辅助变量 $z = A x - b$,则原问题等价于
\[
\min_{x, z} \; \frac{1}{2} \|z\|_F^2 + \mu \|x\|_{1,2} \quad \text{s.t.} \quad z = A x - b.
\]
构造拉格朗日函数(对偶变量为 $y \in \mathbb{R}^{m \times l}$):
\[
\mathcal{L}(x, z; y) = \frac{1}{2} \|z\|_F^2 + \mu \|x\|_{1,2} + \langle y, A x - b - z \rangle,
\]
并对 $(z,x)$ 求极小:
\begin{itemize}
\item 对 $z$ 最小化:  
  $\min_z \left\{ \frac{1}{2} \|z\|_F^2 - \langle y, z \rangle \right\}$  
  在 $z = y$ 处取得最小值,代入得 $-\frac{1}{2} \|y\|_F^2$;
\item 对 $x$ 最小化:  
  $\min_x \left\{ \mu \|x\|_{1,2} + \langle A^\top y, x \rangle \right\} = 
  \begin{cases}
  0, & \text{if } \|(A^\top y)_{i,:}\|_2 \leq \mu,\ \forall i, \\
  -\infty, & \text{otherwise}.
  \end{cases}$
\end{itemize}
将上述结果代入拉格朗日对偶函数可得
\[
g(y) = \inf_{x,z} \mathcal{L}(x,z;y) = 
\begin{cases}
-\dfrac{1}{2} \|y\|_F^2 - \langle y, b \rangle, & \text{if } \|(A^\top y)_{i,:}\|_2 \leq \mu,\ \forall i, \\
-\infty, & \text{otherwise}.
\end{cases}
\]
因此,(\ref{eq:prob}) 的对偶问题为:
\begin{equation*}
    \max_{y \in \mathbb{R}^{m \times l}} \; -\frac{1}{2} \|y\|_F^2 - \langle y, b \rangle
    \quad \text{s.t.} \quad \|(A^\top y)_{i,:}\|_2 \leq \mu, \quad i = 1,\dots,n.
\end{equation*}
等价地,可写为最小化形式:
\begin{equation}\label{eq:dual}
    \min_{y\in \mathbb{R}^{m \times l}} \; \frac{1}{2} \|y\|_F^2 + \langle y, b \rangle
    \quad \text{s.t.} \quad \|(A^\top y)_{i,:}\|_2 \leq \mu, \quad i = 1,\dots,n.
\end{equation}

\subsection{次梯度算法}
为使用次梯度法(Subgradient Descent)求解,我们把正则项 $\|x\|_{1,2}$ 的次梯度取为:
\begin{equation}\label{eq:sg}
    {[\partial\|x\|_{1,2}]_{i,:}} = \begin{cases} 
\dfrac{x_{i,:}}{\|x_{i,:}\|_2} & \text{if } \|x_{i,:}\|_2 > \epsilon=10^{-6}, \\
\dfrac{x_{i,:}}{1+\|x_{i,:}\|_2} & \text{otherwise}.
\end{cases}
\end{equation}

我们使用连续化策略,采用外层循环调整参数、内层执行次梯度更新的双层循环策略。对于内层循环,我们使用前期固定、后期消失的步长 $\alpha_j = \min(\alpha_{\max}, \frac{\alpha}{j+1})$,并在相邻两次 $f_{\mu_t}$ 差值小于 $tol$ 时停止内循环。每次内循环结束,我们缩小参数 $tol$ 和 $\alpha_{\max}$。经尝试,选取第 $i$ 轮外循环(设最多 $N$ 轮)的 $\mu_t$ 为
\begin{align}\label{eq:cos}
    \mu_t &= \text{cos\_annealing}(i,N,\mu,\mu_{\max}) \nonumber\\
    &:= \mu + \frac{\mu_{\max}-\mu}{2}\left(1+\cos\left(\frac{i}{N}\pi\right)\right).
\end{align}

算法~\ref{al:SGD} 描述了 SGD 的更新过程。我们取 $\mu_{\max}=2.0$,内循环最大步数 $N_{\text{inner}}$ 为 500(但当 $\mu_t=\mu$ 时允许迭代 5000 步),外循环最大步数 $N_{\text{outer}}$ 为 10,$\alpha_{\max}$ 初值为 0.001,$\alpha=0.5$,$tol$ 初值为 $10^{-3}$。其它细节详见代码。

从图~\ref{fig:all} 和表~\ref{table:all} 中可以看出次梯度法收敛较慢,且最终得到的解稀疏性较差。

\begin{algorithm}[htbp]
\caption{Group LASSO 随机次梯度下降算法(SGD)}
\label{al:SGD}
\begin{algorithmic}[1]
\Require 初始点 $x_0$, 矩阵 $A$, 观测值 $b$, 正则参数 $\mu$
\Ensure 最优解 $x_{\text{opt}}$, 迭代次数, 函数值序列

\State 初始化 $x \gets x_0$, $x_{\text{opt}} \gets x_0$, $f_{\text{best}} \gets f(x_0)$, $k \gets 0$
\For{$\text{out\_iter} = 0$ to $N_{\text{outer}} - 1$}
    \State $\mu_t \gets \text{cos\_annealing}(out\_iter,N_{\text{outer}},\mu,\mu_{\max}) $
    \If{$\text{out\_iter} == N_{\text{outer}} - 1$}
        \State $flag \gets \text{True}$, $tol \gets 10^{-7}$, $N_{\text{inner}} \gets 5000$
    \Else
        \State $flag \gets \text{False}$, $N_{\text{inner}} \gets 500$
    \EndIf
    \For{$j = 0$ to $N_{\text{inner}} - 1$}
        \State 计算当前目标值 $f^k \gets f(x)$
        \If{$f^k < f_{\text{best}}$}
            \State $x_{\text{opt}} \gets x$, $f_{\text{best}} \gets f^k$
        \EndIf
        \If{$j \geq 1$ \textbf{and} $|f_{\mu_t}(x) - f_{\mu_t}(x_{\text{prev}})| < tol$}
            \State \textbf{break}
        \EndIf
        \State $x_{\text{prev}} \gets x$
        \State 计算次梯度 $g_k \gets A^\top (A x - b) + \mu_t \cdot \partial \|x\|_{1,2}$ \Comment{见式~\eqref{eq:sg}}
        \State 计算步长 $\alpha_j \gets \min(\alpha_{\max}, \dfrac{\alpha}{j + 1})$
        \State 更新 $x \gets x - \alpha_j g_k$
        \State $k \gets k + 1$
        \If{$k \geq 5000$}
            \State $flag \gets \text{True}$
            \State \textbf{break}
        \EndIf
    \EndFor
    \If{$flag$}
        \State \textbf{break}
    \EndIf
    \State 更新参数: $tol \gets \max(tol \cdot 0.8, 10^{-6})$, $\alpha_{\max} \gets \alpha_{\max} \cdot 0.9$
\EndFor
\State \Return $x_{\text{opt}}$, $k$, 函数值序列$\{f^k\}$
\end{algorithmic}
\end{algorithm}

\subsection{光滑化梯度法}
我们考虑对正则项 $\|x\|_{1,2}$ 做光滑化后,取梯度为:
\begin{equation}\label{eq:smooth-g}
    [\nabla g(x,\varepsilon)]_{i,:} = \begin{cases} 
\dfrac{x_{i,:}}{\|x_{i,:}\|_2} & \text{if } \|x_{i,:}\|_2 > \epsilon, \\
\dfrac{x_{i,:}}{\epsilon} & \text{otherwise}.
\end{cases}
\end{equation}
光滑化梯度法的迭代过程类似于次梯度法,但在部分细节上有区别:
\begin{itemize}
    \item $tol$ 和 $\mu_t$ 每次内循环开始时缩小到原来的 0.1,其中 $\mu_t$ 控制为不小于 $\mu$。$\mu_t$ 的初值选为 200。
    \item 采用式~\ref{eq:smooth-g} 的光滑化梯度替代次梯度,其中 $\epsilon = \epsilon_t = 10^{-3}\mu_t$。此时子问题的梯度 Lipschitz 常数可选为 $L_t = \|A\|_2^2 + \frac{\mu_t}{\epsilon_t} = \|A\|_2^2 + 10^3$。因此所有内循环均固定步长为 $\frac{1}{L_t}$。
    \item 超参数变化:设置所有内循环总数上限为 5000,每次内循环不超过 500 次。设置 $tol$ 初值为 0.01。
\end{itemize}

从表~\ref{table:all} 中可以看出光滑化梯度法尽管收敛比次梯度法快,但最终得到的解稀疏性更差,这可能是光滑化处理所导致的。

\subsection{近端点梯度算法及其 Nesterov 加速}
在本项目求解的问题形式下,近似点算子可以逐行计算,解析地写出。记 $h(x)=\mu \|x\|_{1,2}$,则
\begin{equation}\label{eq:prox}
    [\operatorname{prox}_{t h}(z)]_{i,:} = \begin{cases}
\left(1 - \dfrac{t\mu}{\|z_{i,:}\|_2} \right) z_{i,:}, & \text{if } \|z_{i,:}\|_2 > t\mu, \\
0, & \text{otherwise}.
\end{cases}
\end{equation}
对于近端点梯度算法(ProxGD)我们仍然采用连续化策略:
\begin{itemize}
    \item 每次内循环做更新:$z^{k+1} = x^k - t A^\top(Ax^k - b)$, $x^{k+1} = \operatorname{prox}_{t h}(z^{k+1})$,其中固定步长 $t = \frac{1}{\|A\|_2^2}$。
    \item $tol$ 和 $\mu_t$ 每次内循环开始时缩小到原来的 0.1,其中 $\mu_t$ 控制为不小于 $\mu$。$\mu_t$ 的初值选为 200。
    \item 超参数:设置所有内循环总数上限为 5000,每次内循环不超过 500 次。设置 $tol$ 初值为 0.01。
\end{itemize}

Nesterov 加速是一种常基于动量的加速方法。这里我们选取著名的 FISTA 加速算法做更新,得到 FProxGD 的迭代格式:
\begin{align*}
    y^k &= x^{k-1} + \frac{k-2}{k+1}(x^{k-1} - x^{k-2}),\\
    x^k &= \operatorname{prox}_{t h}(y^k - t A^\top(A y^k - b)),
\end{align*}
其中固定步长 $t$ 仍取 $\frac{1}{\|A\|_2^2}$。

从图~\ref{fig:all} 和表~\ref{table:all} 中可以看出,ProxGD 和 FProxGD 的各方面性能均优于 SGD 和 GD。两者相比,加速后的 FProxGD 算法在各方面性能均优于 ProxGD。

\subsection{增广拉格朗日法和 ADMM 求解对偶问题}
对于原问题的对偶问题~\ref{eq:dual},我们引入变量 $z = A^\top y$,将对偶问题化为:
\begin{align}\label{eq:dual2}
    \min_{y\in \mathbb{R}^{m \times l}}& \quad \frac{1}{2} \|y\|_F^2 + \langle y, b \rangle \\
    \text{s.t.}& \quad A^\top y + z = 0, \quad \|z_{i,:}\|_2 \leq \mu, \quad i = 1,\dots,n.
\end{align}
我们对约束 $A^\top y + z = 0$ 引入乘子 $\lambda$,定义增广拉格朗日函数
\begin{equation}\label{eq:Lag}
    \mathcal{L}_{\rho}(y,z,\lambda) =  \frac{1}{2} \|y\|_F^2 + \langle y, b \rangle + \langle\lambda, A^\top y + z\rangle + \frac{\rho}{2}\|A^\top y + z\|_F^2.
\end{equation}
我们固定 $\mu$(即不再采用连续化策略),对于每个子问题,以上一个子问题的 $(y,z)$ 为初值求解
\[
(y,z) = \arg\min_{\|z_{i,:}\|_2 \leq \mu}  \mathcal{L}_{\rho_k}(y,z,\lambda^k).
\]
求解方式为固定 $z$,优化 $y$,此时有解析解:
\begin{equation}\label{eq:y-update}
    y = (I + \rho_k A A^\top)^{-1} ( - A \lambda^k + \rho_k A (-z) - b ).
\end{equation}
再将 $y$ 固定,优化 $z$,此时即为做投影:
\begin{equation}\label{eq:z-update}
    z = \operatorname{proj}_{\{\|z_{i,:}\|_2 \leq \mu\}} \left( -\frac{1}{\rho_k}\lambda^k - A^\top y \right).
\end{equation}
ALM 算法交替进行~\ref{eq:y-update} 式和~\ref{eq:z-update} 式,直到 $z$ 的变化值小于 $tol_{\text{inner}}$。接着做乘子更新:
\begin{equation}\label{eq:lambda-update}
    \lambda^{k+1} = \lambda^k + \rho_k (A^\top y + z).
\end{equation}
我们选取初值 $tol_{\text{inner}} = 10^{-6}$, $\rho_1 = 10$,并在每次外循环结束后增大 $\rho_{k+1} = 2\rho_k$,将 $tol_{\text{inner}}$ 减半。内外循环上限分别为 200 轮、20 轮,相邻两次外循环的目标函数值差小于 $10^{-7}$ 则停止循环。

与 ALM 类似,ADMM 求解对偶问题也使用增广拉格朗日函数,区别在于不区分内外循环,而是交替进行~\ref{eq:y-update}、~\ref{eq:z-update} 和~\ref{eq:lambda-update} 的更新。此时,$\rho$ 固定为 100,终止条件设最近 20 轮内目标函数值最大值和最小值差小于 $10^{-6}$ 或运行 1000 轮则停止循环。

事实上,上述增广拉格朗日方程~\ref{eq:Lag} 中的乘子 $\lambda$ 就对应原始变量 $x$。因此我们可以根据 $\lambda$ 的更新作为 $x$ 的更新。从表~\ref{table:all} 中可以看出,ADMM-dual 的求解效率相当高,且求得的解具有很好的稀疏性。

\subsection{ADMM 求解原始问题}
类似地,我们可以写出原始问题求解的 ADMM 算法:
\begin{align*}
    x^{k+1} &= (\rho I + A^\top A)^{-1}(A^\top b + \rho z^k - y^k), \\
    z^{k+1} &= \operatorname{proj}_{\{\|z_{i,:}\|_2 \leq \mu/\rho\}} \left( x^{k+1} + \frac{1}{\rho} y^k \right), \\
    y^{k+1} &= y^k + \rho (x^{k+1} - z^{k+1}).
\end{align*}
此时,$\rho$ 固定为 0.5,终止条件设最近 20 轮内目标函数值最大值和最小值差小于 $10^{-6}$ 或运行 3000 轮则停止循环。然而,从表~\ref{table:all} 和图~\ref{fig:all} 中可以看出,ADMM-primal 的求解效率和精度都不如 ADMM-dual。

\section{总结与说明}

\textbf{声明:} 本项目的代码和本报告的书写有借助 Deepseek、Qwen3-max 两个 AI 工具,具体使用场景包括询问其商用求解器调用方法、优化代码结构、搭建本实验报告的基础 LaTeX 框架等。如上述算法描述有细节缺失,请参考提交代码。

本文围绕 Group Lasso 优化问题,系统实现了包括商业求解器调用、一阶优化算法(次梯度法、光滑化梯度法、近端梯度法及其 Nesterov 加速版本)、对偶方法(增广拉格朗日法 ALM 和 ADMM)以及原始问题的 ADMM 算法在内的多种求解策略,并在统一的测试样例下进行了全面性能评估。

实验结果表明:
\begin{itemize}
\item 商业求解器 MOSEK、Gurobi 在精度上具有显著优势,尤其在直接建模为二阶锥规划时效率远高于通过 CVXPY 接口调用,说明了问题结构利用的重要性。
\item 次梯度法与光滑化梯度法格式简单,但即使精心选取参数和连续化策略仍收敛缓慢,获得的解稀疏性较差,说明这两种方法对于 Group Lasso 优化问题适用性差。
\item ProxGD 及其加速版本 FProxGD 表现出极优的综合性能:收敛速度快、解的质量高、稀疏性良好。
\item 对偶方法(ALM-dual 与 ADMM-dual)展现出极高的计算效率和良好的稀疏恢复能力。特别是 ADMM-dual 仅需 70 次迭代、0.08 秒即可获得接近最优的解,说明将复杂非光滑项移至约束侧后,投影操作变得简洁高效。
\item 相比之下,ADMM-primal 在实际运行中收敛较慢、稀疏性控制不佳(Sparsity 高达 0.6055)。
\end{itemize}

综上所述,对于中等规模的 Group Lasso 问题,若追求高精度解,应当使用直接调用 MOSEK 的 SOCP 方式;若注重算法实现灵活性,FProxGD 或者 ADMM-dual 是较为高效的算法方案。

\end{document}